\chapter[\protect\module{pygsl.deriv} --- NumericalDifferentiation]%
{\protect\module{pygsl.deriv} \\ Numerical Differentiation}
\label{cha:diff-module}

\declaremodule{extension}{pygsl.deriv}%
 \moduleauthor{Pierre  Schnizer}{schnizer@users.sourceforge.net}%
 \modulesynopsis{Numerical  Differentiation}%

\begin{quote}
  This chapter describes the available functions for numerical differentiation.
\end{quote}

The functions described in this chapter compute numerical derivatives by finite
differencing.  An adaptive algorithm is used to find the best choice of finite
difference and to estimate the error in the derivative. This module supersedes
the diff module which has been deprecated with the release of GSL-1. XXX


\begin{funcdesc}{central}{func, x, h}
  This function computes the numerical derivative of the function \var{func} at
  the point \var{x} using an adaptive central difference algorithm with a step
  size of h.  A tuple \code{(result, error)} is returned with the derivative
  and its estimated absolute error.
\end{funcdesc}

\begin{funcdesc}{backward}{func, x, h}
  This function computes the numerical derivative of the function \var{func} at
  the point \var{x} using an adaptive backward difference algorithm with a step
  size of h.  The function \var{func} is evaluated only at points smaller than
  \var{x} and at \var{x} itself.  A tuple \code{(result, error)} is returned
  with the derivative and its estimated absolute error.
\end{funcdesc}

\begin{funcdesc}{forward}{func, x, h}
  This function computes the numerical derivative of the function \var{func} at
  the point \var{x} using an adaptive forward difference algorithm with a step
  size of h.  The function \var{func} is evaluated only at points greater than
  \var{x} and at \var{x} itself.  A tuple \code{(result, error)} is returned
  with the derivative and its estimated absolute error.
\end{funcdesc}


\begin{seealso}
  The algorithms used by these functions are described in the following book:
  \seetext{S.D.\ Conte and Carl de Boor, \emph{Elementary Numerical Analysis:
      An Algorithmic Approach}, McGraw-Hill, 1972.}
\end{seealso}



%% Local Variables:
%% mode: LaTeX
%% mode: auto-fill
%% fill-column: 79
%% indent-tabs-mode: nil
%% ispell-dictionary: "british"
%% reftex-fref-is-default: nil
%% TeX-auto-save: t
%% TeX-command-default: "pdfeLaTeX"
%% TeX-master: "pygsl"
%% TeX-parse-self: t
%% End:
