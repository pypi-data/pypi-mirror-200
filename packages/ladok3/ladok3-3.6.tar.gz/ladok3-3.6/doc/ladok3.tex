\documentclass[a4paper,oneside]{book}
\newenvironment{abstract}{}{}
\usepackage{abstract}
\usepackage{noweb}
% Needed to relax penalty for breaking code chunks across pages, otherwise 
% there might be a lot of space following a code chunk.
\def\nwendcode{\endtrivlist \endgroup}
\let\nwdocspar=\smallbreak

\usepackage[hyphens]{url}
\usepackage{hyperref}
\usepackage{authblk}

\usepackage[utf8]{inputenc}
\usepackage[T1]{fontenc}
\usepackage[swedish,british]{babel}
\usepackage{booktabs}

\usepackage[natbib,style=alphabetic,maxbibnames=99]{biblatex}
\addbibresource{ladok_lib.bib}

\usepackage[inline]{enumitem}
\setlist[enumerate]{label=(\arabic*)}

\usepackage[all]{foreign}
\renewcommand{\foreignfullfont}{}
\renewcommand{\foreignabbrfont}{}

\usepackage{newclude}
\usepackage{import}

\usepackage[strict]{csquotes}
\usepackage[single]{acro}

\usepackage{subcaption}

\usepackage[noend]{algpseudocode}
\usepackage{xparse}

\let\email\texttt

\usepackage[outputdir=ltxobj]{minted}
\setminted{autogobble,linenos}

\usepackage{pythontex}
\setpythontexoutputdir{.}
\setpythontexworkingdir{..}

\usepackage{fancyvrb}

\usepackage{amsmath}
\usepackage{amssymb}
\usepackage{mathtools}
\usepackage{amsthm}
\usepackage{thmtools}
\usepackage[unq]{unique}
\DeclareMathOperator{\powerset}{\mathcal{P}}

\usepackage[binary-units]{siunitx}

\usepackage{pgf}

\usepackage[capitalize]{cleveref}
\crefalias{question}{item}
\crefname{question}{question}{questions}
\Crefname{question}{Question}{Questions}
\creflabelformat{question}{#2\textup{#1}#3}



\title{%
  A LADOK3 Python API
}
\author{%
  Alexander Baltatzis,
  Daniel Bosk,
  Gerald Q.\ \enquote{Chip} Maguire Jr.
}
\affil{%
  KTH EECS\\
  \texttt{\{alba,dbosk,maguire\}@kth.se}
}

\begin{document}
\frontmatter
\maketitle

\vspace*{\fill}
\VerbatimInput{../LICENSE}
\clearpage

\begin{abstract}
  Carrying the one is a fundamental concept in algebra.
We exhibit its origins, its modern use as well as best practices in teaching.

\end{abstract}
\clearpage

\tableofcontents
\clearpage

\mainmatter
\chapter{Introduction}

LADOK (abbreviation of \foreignlanguage{swedish}{Lokalt Adb–baserat 
DOKumentationssystem}, in Swedish) is the national documentation system for 
higher education in Sweden.
This is the documented source code of \texttt{ladok3}, a LADOK3 API wrapper for 
Python.

The \texttt{ladok3} library provides a non-GUI application that, similar to 
access via a web browser, only provides the user with access to the LADOK data 
and functionality that they would actually have based on their specific user 
permissions in LADOK.
It can be seem as a very streamlined web browser just for LADOK's web 
interface.
While the library exploits caching to reduce the load on the LADOK server, this 
represents a subset of the information that would otherwise be obtained via
LADOK's web GUI export functions.

You can install the \texttt{ladok3} package by running
\begin{minted}{bash}
pip3 install ladok3
\end{minted}
in the terminal.
You can find a quick reference by running
\begin{minted}[firstnumber=last]{bash}
pydoc ladok3
\end{minted}

We provide the main class \texttt{LadokSession} (\cref{LadokSession}).
The \texttt{LadokSession} class acts like \enquote{factories} and will return 
objects representing various LADOK data.
These data objects' classes inherit the \texttt{LadokData} (\cref{LadokData}) 
and \texttt{LadokRemoteData} (\cref{LadokRemoteData}) classes.
Data of the type \texttt{LadokData} is not expected to change, unlike 
\texttt{LadokRemoteData}, which is.
Objects of type \texttt{LadokRemoteData} know how to update themselves, \ie fetch 
and refresh their data from LADOK.
When relevant they can also write data to LADOK, \ie update entries such as 
results.

One design criteria is to improve performance.
We do this by caching all factory methods of any \texttt{LadokSession}.
The \texttt{LadokSession} itself is also designed to be cacheable: if the session to 
LADOK expires, it will automatically reauthenticate to establish a new session.



\part{The library}

\documentclass[a4paper,oneside]{book}
\newenvironment{abstract}{}{}
\usepackage{abstract}
\usepackage{noweb}
% Needed to relax penalty for breaking code chunks across pages, otherwise 
% there might be a lot of space following a code chunk.
\def\nwendcode{\endtrivlist \endgroup}
\let\nwdocspar=\smallbreak

\usepackage[hyphens]{url}
\usepackage{hyperref}
\usepackage{authblk}

\usepackage[utf8]{inputenc}
\usepackage[T1]{fontenc}
\usepackage[swedish,british]{babel}
\usepackage{booktabs}

\usepackage[natbib,style=alphabetic,maxbibnames=99]{biblatex}
\addbibresource{ladok_lib.bib}

\usepackage[inline]{enumitem}
\setlist[enumerate]{label=(\arabic*)}

\usepackage[all]{foreign}
\renewcommand{\foreignfullfont}{}
\renewcommand{\foreignabbrfont}{}

\usepackage{newclude}
\usepackage{import}

\usepackage[strict]{csquotes}
\usepackage[single]{acro}

\usepackage{subcaption}

\usepackage[noend]{algpseudocode}
\usepackage{xparse}

\let\email\texttt

\usepackage[outputdir=ltxobj]{minted}
\setminted{autogobble,linenos}

\usepackage{pythontex}
\setpythontexoutputdir{.}
\setpythontexworkingdir{..}

\usepackage{fancyvrb}

\usepackage{amsmath}
\usepackage{amssymb}
\usepackage{mathtools}
\usepackage{amsthm}
\usepackage{thmtools}
\usepackage[unq]{unique}
\DeclareMathOperator{\powerset}{\mathcal{P}}

\usepackage[binary-units]{siunitx}

\usepackage{pgf}

\usepackage[capitalize]{cleveref}
\crefalias{question}{item}
\crefname{question}{question}{questions}
\Crefname{question}{Question}{Questions}
\creflabelformat{question}{#2\textup{#1}#3}



\title{%
  A LADOK3 Python API
}
\author{%
  Alexander Baltatzis,
  Daniel Bosk,
  Gerald Q.\ \enquote{Chip} Maguire Jr.
}
\affil{%
  KTH EECS\\
  \texttt{\{alba,dbosk,maguire\}@kth.se}
}

\begin{document}
\frontmatter
\maketitle

\vspace*{\fill}
\VerbatimInput{../LICENSE}
\clearpage

\begin{abstract}
  Carrying the one is a fundamental concept in algebra.
We exhibit its origins, its modern use as well as best practices in teaching.

\end{abstract}
\clearpage

\tableofcontents
\clearpage

\mainmatter
\chapter{Introduction}

LADOK (abbreviation of \foreignlanguage{swedish}{Lokalt Adb–baserat 
DOKumentationssystem}, in Swedish) is the national documentation system for 
higher education in Sweden.
This is the documented source code of \texttt{ladok3}, a LADOK3 API wrapper for 
Python.

The \texttt{ladok3} library provides a non-GUI application that, similar to 
access via a web browser, only provides the user with access to the LADOK data 
and functionality that they would actually have based on their specific user 
permissions in LADOK.
It can be seem as a very streamlined web browser just for LADOK's web 
interface.
While the library exploits caching to reduce the load on the LADOK server, this 
represents a subset of the information that would otherwise be obtained via
LADOK's web GUI export functions.

You can install the \texttt{ladok3} package by running
\begin{minted}{bash}
pip3 install ladok3
\end{minted}
in the terminal.
You can find a quick reference by running
\begin{minted}[firstnumber=last]{bash}
pydoc ladok3
\end{minted}

We provide the main class \texttt{LadokSession} (\cref{LadokSession}).
The \texttt{LadokSession} class acts like \enquote{factories} and will return 
objects representing various LADOK data.
These data objects' classes inherit the \texttt{LadokData} (\cref{LadokData}) 
and \texttt{LadokRemoteData} (\cref{LadokRemoteData}) classes.
Data of the type \texttt{LadokData} is not expected to change, unlike 
\texttt{LadokRemoteData}, which is.
Objects of type \texttt{LadokRemoteData} know how to update themselves, \ie fetch 
and refresh their data from LADOK.
When relevant they can also write data to LADOK, \ie update entries such as 
results.

One design criteria is to improve performance.
We do this by caching all factory methods of any \texttt{LadokSession}.
The \texttt{LadokSession} itself is also designed to be cacheable: if the session to 
LADOK expires, it will automatically reauthenticate to establish a new session.



\part{The library}

\documentclass[a4paper,oneside]{book}
\newenvironment{abstract}{}{}
\usepackage{abstract}
\usepackage{noweb}
% Needed to relax penalty for breaking code chunks across pages, otherwise 
% there might be a lot of space following a code chunk.
\def\nwendcode{\endtrivlist \endgroup}
\let\nwdocspar=\smallbreak

\usepackage[hyphens]{url}
\usepackage{hyperref}
\usepackage{authblk}

\usepackage[utf8]{inputenc}
\usepackage[T1]{fontenc}
\usepackage[swedish,british]{babel}
\usepackage{booktabs}

\usepackage[natbib,style=alphabetic,maxbibnames=99]{biblatex}
\addbibresource{ladok_lib.bib}

\usepackage[inline]{enumitem}
\setlist[enumerate]{label=(\arabic*)}

\usepackage[all]{foreign}
\renewcommand{\foreignfullfont}{}
\renewcommand{\foreignabbrfont}{}

\usepackage{newclude}
\usepackage{import}

\usepackage[strict]{csquotes}
\usepackage[single]{acro}

\usepackage{subcaption}

\usepackage[noend]{algpseudocode}
\usepackage{xparse}

\let\email\texttt

\usepackage[outputdir=ltxobj]{minted}
\setminted{autogobble,linenos}

\usepackage{pythontex}
\setpythontexoutputdir{.}
\setpythontexworkingdir{..}

\usepackage{fancyvrb}

\usepackage{amsmath}
\usepackage{amssymb}
\usepackage{mathtools}
\usepackage{amsthm}
\usepackage{thmtools}
\usepackage[unq]{unique}
\DeclareMathOperator{\powerset}{\mathcal{P}}

\usepackage[binary-units]{siunitx}

\usepackage{pgf}

\usepackage[capitalize]{cleveref}
\crefalias{question}{item}
\crefname{question}{question}{questions}
\Crefname{question}{Question}{Questions}
\creflabelformat{question}{#2\textup{#1}#3}



\title{%
  A LADOK3 Python API
}
\author{%
  Alexander Baltatzis,
  Daniel Bosk,
  Gerald Q.\ \enquote{Chip} Maguire Jr.
}
\affil{%
  KTH EECS\\
  \texttt{\{alba,dbosk,maguire\}@kth.se}
}

\begin{document}
\frontmatter
\maketitle

\vspace*{\fill}
\VerbatimInput{../LICENSE}
\clearpage

\begin{abstract}
  Carrying the one is a fundamental concept in algebra.
We exhibit its origins, its modern use as well as best practices in teaching.

\end{abstract}
\clearpage

\tableofcontents
\clearpage

\mainmatter
\chapter{Introduction}

LADOK (abbreviation of \foreignlanguage{swedish}{Lokalt Adb–baserat 
DOKumentationssystem}, in Swedish) is the national documentation system for 
higher education in Sweden.
This is the documented source code of \texttt{ladok3}, a LADOK3 API wrapper for 
Python.

The \texttt{ladok3} library provides a non-GUI application that, similar to 
access via a web browser, only provides the user with access to the LADOK data 
and functionality that they would actually have based on their specific user 
permissions in LADOK.
It can be seem as a very streamlined web browser just for LADOK's web 
interface.
While the library exploits caching to reduce the load on the LADOK server, this 
represents a subset of the information that would otherwise be obtained via
LADOK's web GUI export functions.

You can install the \texttt{ladok3} package by running
\begin{minted}{bash}
pip3 install ladok3
\end{minted}
in the terminal.
You can find a quick reference by running
\begin{minted}[firstnumber=last]{bash}
pydoc ladok3
\end{minted}

We provide the main class \texttt{LadokSession} (\cref{LadokSession}).
The \texttt{LadokSession} class acts like \enquote{factories} and will return 
objects representing various LADOK data.
These data objects' classes inherit the \texttt{LadokData} (\cref{LadokData}) 
and \texttt{LadokRemoteData} (\cref{LadokRemoteData}) classes.
Data of the type \texttt{LadokData} is not expected to change, unlike 
\texttt{LadokRemoteData}, which is.
Objects of type \texttt{LadokRemoteData} know how to update themselves, \ie fetch 
and refresh their data from LADOK.
When relevant they can also write data to LADOK, \ie update entries such as 
results.

One design criteria is to improve performance.
We do this by caching all factory methods of any \texttt{LadokSession}.
The \texttt{LadokSession} itself is also designed to be cacheable: if the session to 
LADOK expires, it will automatically reauthenticate to establish a new session.



\part{The library}

\documentclass[a4paper,oneside]{book}
\newenvironment{abstract}{}{}
\usepackage{abstract}
\usepackage{noweb}
% Needed to relax penalty for breaking code chunks across pages, otherwise 
% there might be a lot of space following a code chunk.
\def\nwendcode{\endtrivlist \endgroup}
\let\nwdocspar=\smallbreak

\usepackage[hyphens]{url}
\usepackage{hyperref}
\usepackage{authblk}

\input{preamble.tex}

\title{%
  A LADOK3 Python API
}
\author{%
  Alexander Baltatzis,
  Daniel Bosk,
  Gerald Q.\ \enquote{Chip} Maguire Jr.
}
\affil{%
  KTH EECS\\
  \texttt{\{alba,dbosk,maguire\}@kth.se}
}

\begin{document}
\frontmatter
\maketitle

\vspace*{\fill}
\VerbatimInput{../LICENSE}
\clearpage

\begin{abstract}
  \input{abstract.tex}
\end{abstract}
\clearpage

\tableofcontents
\clearpage

\mainmatter
\chapter{Introduction}

LADOK (abbreviation of \foreignlanguage{swedish}{Lokalt Adb–baserat 
DOKumentationssystem}, in Swedish) is the national documentation system for 
higher education in Sweden.
This is the documented source code of \texttt{ladok3}, a LADOK3 API wrapper for 
Python.

The \texttt{ladok3} library provides a non-GUI application that, similar to 
access via a web browser, only provides the user with access to the LADOK data 
and functionality that they would actually have based on their specific user 
permissions in LADOK.
It can be seem as a very streamlined web browser just for LADOK's web 
interface.
While the library exploits caching to reduce the load on the LADOK server, this 
represents a subset of the information that would otherwise be obtained via
LADOK's web GUI export functions.

You can install the \texttt{ladok3} package by running
\begin{minted}{bash}
pip3 install ladok3
\end{minted}
in the terminal.
You can find a quick reference by running
\begin{minted}[firstnumber=last]{bash}
pydoc ladok3
\end{minted}

We provide the main class \texttt{LadokSession} (\cref{LadokSession}).
The \texttt{LadokSession} class acts like \enquote{factories} and will return 
objects representing various LADOK data.
These data objects' classes inherit the \texttt{LadokData} (\cref{LadokData}) 
and \texttt{LadokRemoteData} (\cref{LadokRemoteData}) classes.
Data of the type \texttt{LadokData} is not expected to change, unlike 
\texttt{LadokRemoteData}, which is.
Objects of type \texttt{LadokRemoteData} know how to update themselves, \ie fetch 
and refresh their data from LADOK.
When relevant they can also write data to LADOK, \ie update entries such as 
results.

One design criteria is to improve performance.
We do this by caching all factory methods of any \texttt{LadokSession}.
The \texttt{LadokSession} itself is also designed to be cacheable: if the session to 
LADOK expires, it will automatically reauthenticate to establish a new session.



\part{The library}

\input{../src/ladok3/ladok3.tex}


\part{API calls}

\input{../src/ladok3/api.tex}
\input{../src/ladok3/undoc.tex}



\part{A command-line interface}

\chapter{The base interface}

\input{../src/ladok3/cli.tex}

\chapter{The \texttt{data} command}

\input{../src/ladok3/data.tex}

\chapter{The \texttt{report} command}

\input{../src/ladok3/report.tex}

\chapter{The \texttt{student} command}

\input{../src/ladok3/student.tex}



\part{Other example applications}

\chapter{Transfer results from KTH Canvas to LADOK}

Here we provide an example program~\texttt{canvas2ladok.py} which exports 
results from KTH Canvas to LADOK for the introductory programming course~prgi 
(DD1315).
You can find an up-to-date version of this chapter at
\begin{center}
  \url{https://github.com/dbosk/intropy/tree/master/adm/reporting}.
\end{center}

\input{../examples/canvas2ladok.tex}


\backmatter
\printbibliography


\end{document}



\part{API calls}

\input{../src/ladok3/api.tex}
\input{../src/ladok3/undoc.tex}



\part{A command-line interface}

\chapter{The base interface}

\input{../src/ladok3/cli.tex}

\chapter{The \texttt{data} command}

\input{../src/ladok3/data.tex}

\chapter{The \texttt{report} command}

\documentclass[12pt]{scrartcl}%{article}
\usepackage{scrlayer-scrpage}
\pagestyle{scrheadings}
\usepackage{tabularx}
\usepackage{lastpage}
\usepackage{amsmath,amssymb}
\usepackage[yyyymmdd,hhmmss]{datetime}
\usepackage{graphicx}
\usepackage[siunitx, RPvoltages]{circuitikz}

\usepackage[english]{babel}
\usepackage[T1]{fontenc}
\usepackage{lmodern}
\usepackage{csvsimple}
\usepackage[ansinew]{inputenc}  % Unix
  % \usepackage[ansinew]{inputenc}  % Windows
  % \usepackage[applemac]{inputenc} % Mac
  
\usepackage[a4paper, left=2cm, right=2cm, top=3.0cm, bottom = 2.5cm]{geometry}
\setlength{\headsep}{0.5cm}

\ihead{\includegraphics[scale=0.18]{Zahner_Logo.pdf}}

\usepackage{lmodern}
\cfoot{Page \thepage}

\usepackage[thmmarks,amsmath,hyperref,noconfig]{ntheorem}
  
\usepackage[colorlinks,
pdfpagelabels,
pdfstartview = FitH,
bookmarksopen = true,
bookmarksnumbered = true,
linkcolor = black,
plainpages = false,
hypertexnames = false,
citecolor = black] {hyperref}  


\usepackage{color}
\definecolor{white}{rgb}{1,1,1}
\definecolor{darkred}{rgb}{0.3,0,0}
\definecolor{darkgreen}{rgb}{0,0.3,0}
\definecolor{darkblue}{rgb}{0,0,0.3}
\definecolor{pink}{rgb}{0.78,0.09,0.51}
\definecolor{purple}{rgb}{0.28,0.24,0.55}
\definecolor{orange}{rgb}{1,0.6,0.0}
\definecolor{grey}{rgb}{0.4,0.4,0.4}


\DeclareMathOperator{\GL}{GL}
\newcommand{\N}{\mathbb{N}}
\newcommand{\Z}{\mathbb{Z}}
\newcommand{\Q}{\mathbb{Q}}
\newcommand{\R}{\mathbb{R}}
\newcommand{\C}{\mathbb{C}}
\newcommand{\cP}{{\mathcal P}} 

\begin{document}
\vspace*{30mm}
\begin{center}
  \title{Capacitor Test Report}
  \textbf{{\huge Capacitor Test Report}}
\end{center}
\vspace*{40mm}


\section{Informations}
\vspace{5mm}
\begin{center}
  \begin{tabularx}{0.8\textwidth}{| X | X |}
    \hline
                               &                    \\
    \textbf{Test object name}: & \PYVAR{objectname} \\
                               &                    \\
    \hline
                               &                    \\
    \textbf{Date of test}:     & \today             \\
                               &                    \\
    \hline
                               &                    \\
    \textbf{Time of test}:     & \currenttime       \\
                               &                    \\
    \hline
  \end{tabularx}
\end{center}
\newpage

\section{Measurement}

\subsection{CV}
\includegraphics[width=1\textwidth]{ \PYVAR{cv_filename} }

\newpage

\subsection{EIS}
\includegraphics[width=1\textwidth]{ \PYVAR{eis_filename} }

\begin{center}
  \begin{tabular}{c|c|c}
    \bfseries Frequency $[Hz]$ & \bfseries |Impedance| $[\Omega]$ & \bfseries Phase [\textdegree]
    \csvreader[head to column names, /csv/separator=semicolon]{ \PYVAR{eis_csv_filename} }{}
    {                                                                                             \\\hline \Frequency & \Impedance & \Phase}
  \end{tabular}
\end{center}

\section{Evaluation}

\subsection{CV}

The basic equation for the relationship between voltage, capacitance and charge is:

\begin{center}
  \begin{math}
    C = \dfrac{Q}{U}
  \end{math}
  or
  \begin{math}
    C = \dfrac{I * t}{U}
  \end{math}
\end{center}
\noindent
Where C = capacity, Q = charge, I = current, t = time and U = voltage. If we assume a linearly increasing or decreasing voltage (dU/dt - as in CV) as well as a voltage of zero volts at the beginning of the experiment, we get the following equation:

\begin{center}
  \begin{math}
    C = \dfrac{I}{\frac{dU}{dt}} = \dfrac{\dfrac{\PYVAR{cv_maximum_current} - (\PYVAR{cv_minimum_current})}{2}}{\PYVAR{cv_scanrate} \frac{V}{s}} = \PYVAR{capacitance_cv}
  \end{math}
\end{center}

\subsection{EIS}

The parameters of the following simplified equivalent circuit model of a capacitor are fitted with the impedance spectrum:

\begin{center}
  \begin{circuitikz}[european]
    \draw (0,0)
    to[R = $R_0$] (2,0)
    to[C = $C_0$] (4,0)
    to[L = $L_0$] (6,0)
    ;
  \end{circuitikz}
\end{center}

\noindent
Results of fitting the model parameters to the measured data are shown in the table:
\begin{center}
  \begin{tabular}{|c|c|c|}
    \hline
    \textbf{Element} & \textbf{Fitted Value}   & \textbf{Fit Error}              \\
    \hline
    $R_0$            & \PYVAR{resistance_eis}  & \PYVAR{resistance_eis_error}\%  \\
    \hline
    $C_0$            & \PYVAR{capacitance_eis} & \PYVAR{capacitance_eis_error}\% \\
    \hline
    $L_0$            & \PYVAR{inductance_eis}  & \PYVAR{inductance_eis_error}\%  \\
    \hline
  \end{tabular}
\end{center}

\vspace{5mm}
\noindent
In the following figure, the measurement data and the curve for the simulated model with the fitted parameters are shown:
\begin{center}
  \includegraphics[width=1\textwidth]{ \PYVAR{eis_fitted_filename} }
\end{center}

\end{document}

\chapter{The \texttt{student} command}

\input{../src/ladok3/student.tex}



\part{Other example applications}

\chapter{Transfer results from KTH Canvas to LADOK}

Here we provide an example program~\texttt{canvas2ladok.py} which exports 
results from KTH Canvas to LADOK for the introductory programming course~prgi 
(DD1315).
You can find an up-to-date version of this chapter at
\begin{center}
  \url{https://github.com/dbosk/intropy/tree/master/adm/reporting}.
\end{center}

\input{../examples/canvas2ladok.tex}


\backmatter
\printbibliography


\end{document}



\part{API calls}

\input{../src/ladok3/api.tex}
\input{../src/ladok3/undoc.tex}



\part{A command-line interface}

\chapter{The base interface}

\input{../src/ladok3/cli.tex}

\chapter{The \texttt{data} command}

\input{../src/ladok3/data.tex}

\chapter{The \texttt{report} command}

\documentclass[12pt]{scrartcl}%{article}
\usepackage{scrlayer-scrpage}
\pagestyle{scrheadings}
\usepackage{tabularx}
\usepackage{lastpage}
\usepackage{amsmath,amssymb}
\usepackage[yyyymmdd,hhmmss]{datetime}
\usepackage{graphicx}
\usepackage[siunitx, RPvoltages]{circuitikz}

\usepackage[english]{babel}
\usepackage[T1]{fontenc}
\usepackage{lmodern}
\usepackage{csvsimple}
\usepackage[ansinew]{inputenc}  % Unix
  % \usepackage[ansinew]{inputenc}  % Windows
  % \usepackage[applemac]{inputenc} % Mac
  
\usepackage[a4paper, left=2cm, right=2cm, top=3.0cm, bottom = 2.5cm]{geometry}
\setlength{\headsep}{0.5cm}

\ihead{\includegraphics[scale=0.18]{Zahner_Logo.pdf}}

\usepackage{lmodern}
\cfoot{Page \thepage}

\usepackage[thmmarks,amsmath,hyperref,noconfig]{ntheorem}
  
\usepackage[colorlinks,
pdfpagelabels,
pdfstartview = FitH,
bookmarksopen = true,
bookmarksnumbered = true,
linkcolor = black,
plainpages = false,
hypertexnames = false,
citecolor = black] {hyperref}  


\usepackage{color}
\definecolor{white}{rgb}{1,1,1}
\definecolor{darkred}{rgb}{0.3,0,0}
\definecolor{darkgreen}{rgb}{0,0.3,0}
\definecolor{darkblue}{rgb}{0,0,0.3}
\definecolor{pink}{rgb}{0.78,0.09,0.51}
\definecolor{purple}{rgb}{0.28,0.24,0.55}
\definecolor{orange}{rgb}{1,0.6,0.0}
\definecolor{grey}{rgb}{0.4,0.4,0.4}


\DeclareMathOperator{\GL}{GL}
\newcommand{\N}{\mathbb{N}}
\newcommand{\Z}{\mathbb{Z}}
\newcommand{\Q}{\mathbb{Q}}
\newcommand{\R}{\mathbb{R}}
\newcommand{\C}{\mathbb{C}}
\newcommand{\cP}{{\mathcal P}} 

\begin{document}
\vspace*{30mm}
\begin{center}
  \title{Capacitor Test Report}
  \textbf{{\huge Capacitor Test Report}}
\end{center}
\vspace*{40mm}


\section{Informations}
\vspace{5mm}
\begin{center}
  \begin{tabularx}{0.8\textwidth}{| X | X |}
    \hline
                               &                    \\
    \textbf{Test object name}: & \PYVAR{objectname} \\
                               &                    \\
    \hline
                               &                    \\
    \textbf{Date of test}:     & \today             \\
                               &                    \\
    \hline
                               &                    \\
    \textbf{Time of test}:     & \currenttime       \\
                               &                    \\
    \hline
  \end{tabularx}
\end{center}
\newpage

\section{Measurement}

\subsection{CV}
\includegraphics[width=1\textwidth]{ \PYVAR{cv_filename} }

\newpage

\subsection{EIS}
\includegraphics[width=1\textwidth]{ \PYVAR{eis_filename} }

\begin{center}
  \begin{tabular}{c|c|c}
    \bfseries Frequency $[Hz]$ & \bfseries |Impedance| $[\Omega]$ & \bfseries Phase [\textdegree]
    \csvreader[head to column names, /csv/separator=semicolon]{ \PYVAR{eis_csv_filename} }{}
    {                                                                                             \\\hline \Frequency & \Impedance & \Phase}
  \end{tabular}
\end{center}

\section{Evaluation}

\subsection{CV}

The basic equation for the relationship between voltage, capacitance and charge is:

\begin{center}
  \begin{math}
    C = \dfrac{Q}{U}
  \end{math}
  or
  \begin{math}
    C = \dfrac{I * t}{U}
  \end{math}
\end{center}
\noindent
Where C = capacity, Q = charge, I = current, t = time and U = voltage. If we assume a linearly increasing or decreasing voltage (dU/dt - as in CV) as well as a voltage of zero volts at the beginning of the experiment, we get the following equation:

\begin{center}
  \begin{math}
    C = \dfrac{I}{\frac{dU}{dt}} = \dfrac{\dfrac{\PYVAR{cv_maximum_current} - (\PYVAR{cv_minimum_current})}{2}}{\PYVAR{cv_scanrate} \frac{V}{s}} = \PYVAR{capacitance_cv}
  \end{math}
\end{center}

\subsection{EIS}

The parameters of the following simplified equivalent circuit model of a capacitor are fitted with the impedance spectrum:

\begin{center}
  \begin{circuitikz}[european]
    \draw (0,0)
    to[R = $R_0$] (2,0)
    to[C = $C_0$] (4,0)
    to[L = $L_0$] (6,0)
    ;
  \end{circuitikz}
\end{center}

\noindent
Results of fitting the model parameters to the measured data are shown in the table:
\begin{center}
  \begin{tabular}{|c|c|c|}
    \hline
    \textbf{Element} & \textbf{Fitted Value}   & \textbf{Fit Error}              \\
    \hline
    $R_0$            & \PYVAR{resistance_eis}  & \PYVAR{resistance_eis_error}\%  \\
    \hline
    $C_0$            & \PYVAR{capacitance_eis} & \PYVAR{capacitance_eis_error}\% \\
    \hline
    $L_0$            & \PYVAR{inductance_eis}  & \PYVAR{inductance_eis_error}\%  \\
    \hline
  \end{tabular}
\end{center}

\vspace{5mm}
\noindent
In the following figure, the measurement data and the curve for the simulated model with the fitted parameters are shown:
\begin{center}
  \includegraphics[width=1\textwidth]{ \PYVAR{eis_fitted_filename} }
\end{center}

\end{document}

\chapter{The \texttt{student} command}

\input{../src/ladok3/student.tex}



\part{Other example applications}

\chapter{Transfer results from KTH Canvas to LADOK}

Here we provide an example program~\texttt{canvas2ladok.py} which exports 
results from KTH Canvas to LADOK for the introductory programming course~prgi 
(DD1315).
You can find an up-to-date version of this chapter at
\begin{center}
  \url{https://github.com/dbosk/intropy/tree/master/adm/reporting}.
\end{center}

\input{../examples/canvas2ladok.tex}


\backmatter
\printbibliography


\end{document}



\part{API calls}

\input{../src/ladok3/api.tex}
\input{../src/ladok3/undoc.tex}



\part{A command-line interface}

\chapter{The base interface}

\input{../src/ladok3/cli.tex}

\chapter{The \texttt{data} command}

\input{../src/ladok3/data.tex}

\chapter{The \texttt{report} command}

\documentclass[12pt]{scrartcl}%{article}
\usepackage{scrlayer-scrpage}
\pagestyle{scrheadings}
\usepackage{tabularx}
\usepackage{lastpage}
\usepackage{amsmath,amssymb}
\usepackage[yyyymmdd,hhmmss]{datetime}
\usepackage{graphicx}
\usepackage[siunitx, RPvoltages]{circuitikz}

\usepackage[english]{babel}
\usepackage[T1]{fontenc}
\usepackage{lmodern}
\usepackage{csvsimple}
\usepackage[ansinew]{inputenc}  % Unix
  % \usepackage[ansinew]{inputenc}  % Windows
  % \usepackage[applemac]{inputenc} % Mac
  
\usepackage[a4paper, left=2cm, right=2cm, top=3.0cm, bottom = 2.5cm]{geometry}
\setlength{\headsep}{0.5cm}

\ihead{\includegraphics[scale=0.18]{Zahner_Logo.pdf}}

\usepackage{lmodern}
\cfoot{Page \thepage}

\usepackage[thmmarks,amsmath,hyperref,noconfig]{ntheorem}
  
\usepackage[colorlinks,
pdfpagelabels,
pdfstartview = FitH,
bookmarksopen = true,
bookmarksnumbered = true,
linkcolor = black,
plainpages = false,
hypertexnames = false,
citecolor = black] {hyperref}  


\usepackage{color}
\definecolor{white}{rgb}{1,1,1}
\definecolor{darkred}{rgb}{0.3,0,0}
\definecolor{darkgreen}{rgb}{0,0.3,0}
\definecolor{darkblue}{rgb}{0,0,0.3}
\definecolor{pink}{rgb}{0.78,0.09,0.51}
\definecolor{purple}{rgb}{0.28,0.24,0.55}
\definecolor{orange}{rgb}{1,0.6,0.0}
\definecolor{grey}{rgb}{0.4,0.4,0.4}


\DeclareMathOperator{\GL}{GL}
\newcommand{\N}{\mathbb{N}}
\newcommand{\Z}{\mathbb{Z}}
\newcommand{\Q}{\mathbb{Q}}
\newcommand{\R}{\mathbb{R}}
\newcommand{\C}{\mathbb{C}}
\newcommand{\cP}{{\mathcal P}} 

\begin{document}
\vspace*{30mm}
\begin{center}
  \title{Capacitor Test Report}
  \textbf{{\huge Capacitor Test Report}}
\end{center}
\vspace*{40mm}


\section{Informations}
\vspace{5mm}
\begin{center}
  \begin{tabularx}{0.8\textwidth}{| X | X |}
    \hline
                               &                    \\
    \textbf{Test object name}: & \PYVAR{objectname} \\
                               &                    \\
    \hline
                               &                    \\
    \textbf{Date of test}:     & \today             \\
                               &                    \\
    \hline
                               &                    \\
    \textbf{Time of test}:     & \currenttime       \\
                               &                    \\
    \hline
  \end{tabularx}
\end{center}
\newpage

\section{Measurement}

\subsection{CV}
\includegraphics[width=1\textwidth]{ \PYVAR{cv_filename} }

\newpage

\subsection{EIS}
\includegraphics[width=1\textwidth]{ \PYVAR{eis_filename} }

\begin{center}
  \begin{tabular}{c|c|c}
    \bfseries Frequency $[Hz]$ & \bfseries |Impedance| $[\Omega]$ & \bfseries Phase [\textdegree]
    \csvreader[head to column names, /csv/separator=semicolon]{ \PYVAR{eis_csv_filename} }{}
    {                                                                                             \\\hline \Frequency & \Impedance & \Phase}
  \end{tabular}
\end{center}

\section{Evaluation}

\subsection{CV}

The basic equation for the relationship between voltage, capacitance and charge is:

\begin{center}
  \begin{math}
    C = \dfrac{Q}{U}
  \end{math}
  or
  \begin{math}
    C = \dfrac{I * t}{U}
  \end{math}
\end{center}
\noindent
Where C = capacity, Q = charge, I = current, t = time and U = voltage. If we assume a linearly increasing or decreasing voltage (dU/dt - as in CV) as well as a voltage of zero volts at the beginning of the experiment, we get the following equation:

\begin{center}
  \begin{math}
    C = \dfrac{I}{\frac{dU}{dt}} = \dfrac{\dfrac{\PYVAR{cv_maximum_current} - (\PYVAR{cv_minimum_current})}{2}}{\PYVAR{cv_scanrate} \frac{V}{s}} = \PYVAR{capacitance_cv}
  \end{math}
\end{center}

\subsection{EIS}

The parameters of the following simplified equivalent circuit model of a capacitor are fitted with the impedance spectrum:

\begin{center}
  \begin{circuitikz}[european]
    \draw (0,0)
    to[R = $R_0$] (2,0)
    to[C = $C_0$] (4,0)
    to[L = $L_0$] (6,0)
    ;
  \end{circuitikz}
\end{center}

\noindent
Results of fitting the model parameters to the measured data are shown in the table:
\begin{center}
  \begin{tabular}{|c|c|c|}
    \hline
    \textbf{Element} & \textbf{Fitted Value}   & \textbf{Fit Error}              \\
    \hline
    $R_0$            & \PYVAR{resistance_eis}  & \PYVAR{resistance_eis_error}\%  \\
    \hline
    $C_0$            & \PYVAR{capacitance_eis} & \PYVAR{capacitance_eis_error}\% \\
    \hline
    $L_0$            & \PYVAR{inductance_eis}  & \PYVAR{inductance_eis_error}\%  \\
    \hline
  \end{tabular}
\end{center}

\vspace{5mm}
\noindent
In the following figure, the measurement data and the curve for the simulated model with the fitted parameters are shown:
\begin{center}
  \includegraphics[width=1\textwidth]{ \PYVAR{eis_fitted_filename} }
\end{center}

\end{document}

\chapter{The \texttt{student} command}

\input{../src/ladok3/student.tex}



\part{Other example applications}

\chapter{Transfer results from KTH Canvas to LADOK}

Here we provide an example program~\texttt{canvas2ladok.py} which exports 
results from KTH Canvas to LADOK for the introductory programming course~prgi 
(DD1315).
You can find an up-to-date version of this chapter at
\begin{center}
  \url{https://github.com/dbosk/intropy/tree/master/adm/reporting}.
\end{center}

\input{../examples/canvas2ladok.tex}


\backmatter
\printbibliography


\end{document}
